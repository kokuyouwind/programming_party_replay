\documentclass[index]{subfiles}
\pagenumbering{roman}

\begin{document}

  \chapter*{まえがき}
  \addcontentsline{toc}{chapter}{まえがき}
  
\begin{quote}
\noindent
ランダムに決まる仕様に沿って、みんなでコードを書きましょう。 \\
お互いにコードレビューをして、仕様どおりに書けたかチェック! \\
\\
ですがご注意。仕様はどんどん増えていきます。\\
なかにはとんでもない仕様変更があるかも……? 
\end{quote}

\begin{center}
* * *
\end{center}

本書は、プログラミングゲーム「Programming Party」のリプレイ・解説本です。リプレイとは主にテーブルトークRPGで使われる概念で「実際に遊んだ経緯を記録した文章などの媒体」を指します。要は「こんな感じで遊んだよ」というものですね。

この本も、実際に遊んだ内容を記録したものです。Programming Partyはランダムに仕様を決め、それに沿ってプログラミングをするゲームなので、リプレイではゲーム中に実装したコードを掲載・解説しています。

実装に使用した言語はRubyです。とはいえ、オブジェクト指向言語の経験があれば概ね読めるレベルのコードになっているかと思います。またリプレイだけでなく、ゲーム自体の紹介やゲーム制作にまつわるコラムも掲載していますので、Rubyを書いたことがない方でも読んでいただける内容になっていると思います。

技術書と呼ぶにはマニアックな内容ですが、こんな本もあるんだなと楽しんでいただければ幸いです。

\end{document}
