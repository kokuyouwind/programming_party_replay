\usepackage[dvipdfmx]{graphicx}

\usepackage{amsmath,amssymb,amscd}
\usepackage{mathtools}
\usepackage{euler}
\usepackage{okumacro}
\usepackage{tikz}
\usepackage{wrapfig}
\usepackage{setspace}
\usepackage{ascmac}
\usepackage{here}
%%シングルクォート対応
\usepackage{upquote,textcomp}
\xspcode`'=0
% listings
\usepackage{listings,jlisting}
\definecolor{OliveGreen}{cmyk}{0.64,0,0.95,0.40}
\definecolor{colFunc}{rgb}{1,0.07,0.54}
\definecolor{CadetBlue}{cmyk}{0.62,0.57,0.23,0}
\definecolor{Brown}{cmyk}{0,0.81,1,0.60}
\definecolor{colID}{rgb}{0.63,0.44,0}

\lstloadlanguages{Ruby}
\lstset{
language=Ruby,%プログラミング言語によって変える。
basicstyle=\small\ttfamily\color{black},
commentstyle = \small\ttfamily\color{orange},
keywordstyle=\small\bfseries\color{blue},
stringstyle=\small\color{Brown},
morecomment=[f][\color{red}]-,
morecomment=[f][\color{OliveGreen}]+,
tabsize=2,
frame=single,
numbers=none,
breaklines=true,%折り返し
backgroundcolor={\color[gray]{.95}},
captionpos=b,
showstringspaces=false,
xleftmargin=\parindent,
xrightmargin=\parindent,
}
\renewcommand{\lstlistingname}{コード}
% hyperref
%\usepackage[dvipdfmx]{hyperref}
\usepackage{url}
\usepackage{atbegshi}
\ifnum 42146=\euc"A4A2
\AtBeginShipoutFirst{\special{pdf:tounicode EUC-UCS2}}
\else
\AtBeginShipoutFirst{\special{pdf:tounicode 90ms-RKSJ-UCS2}}
\fi
\usetikzlibrary{shapes,backgrounds}

%index
\usepackage{makeidx}
\makeindex
\newcommand{\indexterm}[2]{{\sf #1}\index{#2@#1}}
\newcommand{\term}[2]{#1\index{#2@#1}}
\newcommand{\nindexterm}[4]{{\sf #1#2}\index{#3@#1!#4@---#2}}

\setlength{\textwidth}{\fullwidth}
\setlength{\evensidemargin}{\oddsidemargin}
%デフォルトをゴシックにする
%\renewcommand{\kanjifamilydefault}{\gtdefault}
%各種コマンド
\newcommand{\urlnote}[1]{\footnote{\url{#1}}}
\def\chapterautorefname~#1\null{第~#1章\null}
\newcommand{\refch}[1]{\autoref{ch:#1}}
\def\sectionautorefname~#1\null{~#1節\null}
\newcommand{\refsec}[1]{\autoref{sec:#1}}
\def\lstlistingautorefname~#1\null{コード~#1\null}
\newcommand{\refcode}[1]{\autoref{code:#1}}
\def\figureautorefname~#1\null{図~#1\null}
\newcommand{\reffig}[1]{\autoref{fig:#1}}
%\def\equationautorefname~#1\null{式~#1\null}
%\newcommand{\refeq}[1]{\autoref{eq:#1}}

%表紙・裏表紙の挿入コマンド
\newcommand{\includecover}[1]{
\enlargethispage{\paperwidth}
\thispagestyle{empty}
\vspace*{-1truein}
\vspace*{-\topmargin}
\vspace*{-1.16\headheight}
\vspace*{-\headsep}
\vspace*{-\topskip}
\noindent\hspace*{-1.04in}\hspace*{-\oddsidemargin}
\includegraphics[width=1.005\paperwidth]{#1}
\newpage
}
